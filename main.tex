\documentclass[a4paper, 11pt]{article}

\usepackage[danish]{babel}
\usepackage[utf8]{inputenc}
\usepackage{tgtermes}
\usepackage{fouriernc}
\usepackage[T1]{fontenc}
\usepackage[margin=3cm]{geometry}

\usepackage{amssymb}
\usepackage{amsmath}
\usepackage{amsthm}
\usepackage{multicol}
\usepackage{xcolor}
\usepackage{wrapfig}

\usepackage{enumerate}
\usepackage[shortlabels]{enumitem}
\usepackage{verbatim}
\usepackage{hyperref}
\hypersetup{
    colorlinks=true,
    linkcolor=red,   
    urlcolor=red,
}
\newcommand{\N}{\mathbb{N}}
\newcommand{\Z}{\mathbb{Z}}
\newcommand{\Q}{\mathbb{Q}}
\newcommand{\R}{\mathbb{R}}

 
\title{Projekt Kagedeling\\{\large \textsc{Matematik A}}}
\author{Cecilie Horshauge}
\date{\today}

\begin{document}
\maketitle

\section*{A:}
\textit{Opstil en rekursionsligning, der fastlægger udviklingen i antallet af stykker kage udover samuraimesterens.}\\\\
Reglerne for kagedeling er givet ved:
\begin{itemize}
    \item Alle kagestykker halveres
    \item Samuraimesteren tildeles ét af stykkerne
    \item Denne proces gentages et passende antal gange.
\end{itemize}
Rekursionsligningen \(\)

\section*{B:}
\textit{Bestem ligningens fuldstændige løsning.}\\\\

\section*{C:}
\textit{Bestem de partikulære løsninger med udgangspunkt i begyndelsesværdierne \(y_0=1\), \(y_0=2\), \(y_0=3\) samt \(y_0=s\).}\\\\

\section*{D:}
\textit{Opstil talrækkerne ud fra de partikulære løsninger med begyndelsesværdierne i sp. C og sammenlign svaret med rekursionsligningen fra sp. A.}\\\\

\section*{E:}
\textit{Udvælg en funktionsforskrift og vis hvordan Newtons metode kan anvendes til at bestemme nulpunkt/nulpunkter.}\\\\

\section*{F:}
\textit{Udvælg en differentialligning og vis, hvordan Eulers metode kan anvendes til at bestemme punkter på løsningskurven.}\\\\
\end{document}