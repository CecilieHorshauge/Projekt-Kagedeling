\documentclass[a4paper, 11pt]{article}

\usepackage[danish]{babel}
\usepackage[utf8]{inputenc}
\usepackage{tgtermes}
\usepackage{fouriernc}
\usepackage[T1]{fontenc}
\usepackage[margin=3cm]{geometry}

\usepackage{amssymb}
\usepackage{amsmath}
\usepackage{amsthm}
\usepackage{multicol}
\usepackage{xcolor}
\usepackage{wrapfig}

\usepackage{enumerate}
\usepackage[shortlabels]{enumitem}
\usepackage{verbatim}
\usepackage{hyperref}
\hypersetup{
    colorlinks=true,
    linkcolor=red,   
    urlcolor=red,
}
\newcommand{\N}{\mathbb{N}}
\newcommand{\Z}{\mathbb{Z}}
\newcommand{\Q}{\mathbb{Q}}
\newcommand{\R}{\mathbb{R}}

 
\title{Projekt Kagedeling\\{\large \textsc{Matematik A}}}
\author{Cecilie Horshauge}
\date{\today}

\begin{document}
\maketitle

\section*{A:}
\textit{Opstil en rekursionsligning, der fastlægger udviklingen i antallet af stykker kage udover samuraimesterens.}\\\\
Reglerne for kagedeling er givet ved:
\begin{itemize}
    \item Alle kagestykker halveres
    \item Samuraimesteren tildeles ét af stykkerne
    \item Denne proces gentages et passende antal gange.
\end{itemize}
\textbf{NB: Jeg antager at for hver gentagelse at samuraimesteren får et stykke kage.}\\
Rekursionsligningen \(y_{n+1}=2 \cdot y_n -1\) må lige netop være en passende rekursionsligning. 
Da antallet afhænger af antallet af stykker der var skåret lige inden \(y_n\) og ved \(y_{n+1}\) fordobles antallet af stykker ved at hvert kagestykke halveres. Der tages højde for samuraimesterens kagestykker ved at trække 1 fra.\\
Udviklingen i antal kagestykker udover samuraimesterens kan derfor beskrives med rekursionsligningen
\[y_{n+1}=2 \cdot y_n -1.\]
\clearpage
\section*{B:}
\textit{Bestem ligningens fuldstændige løsning.}\\\\
Rekursionsligningen er inhomogen. Først vil samtlige  løsninger bestemmes med udgangspunkt i sætning 3.\\
Jeg antager først at \(z_n\) er en løsning rekursionsligningen
\[y_{n+1}=2 \cdot y_n -1.\]
Dernæst gætter jeg på at \(z_n=c\), altså at løsningen \(z_n\) er en konstant. Vi kan derfor lave denne manipulation af udtrykket og isolere for c.
\begin{align*}
    z_{n+1}=2 \cdot z_n -1\\
    c=2 \cdot c -1\\
    c=1
\end{align*}
Som følge af sætning 3 bliver den fuldstændige løsning
\[y_{n}=1+k \cdot 2^n\] 
\section*{C:}
\textit{Bestem de partikulære løsninger med udgangspunkt i begyndelsesværdierne \(y_0=1\), \(y_0=2\), \(y_0=3\) samt \(y_0=s\).}\\\\

\section*{D:}
\textit{Opstil talrækkerne ud fra de partikulære løsninger med begyndelsesværdierne i sp. C og sammenlign svaret med rekursionsligningen fra sp. A.}\\\\

\section*{E:}
\textit{Udvælg en funktionsforskrift og vis hvordan Newtons metode kan anvendes til at bestemme nulpunkt/nulpunkter.}\\\\

\section*{F:}
\textit{Udvælg en differentialligning og vis, hvordan Eulers metode kan anvendes til at bestemme punkter på løsningskurven.}\\\\
\end{document}